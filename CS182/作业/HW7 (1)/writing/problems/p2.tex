\item \defpoints{15} [Performing K-Means by Hand]

Let's do $k$-means! To initialize the centroids, we use the $k$-means++ algorithm. And then use Euclidean distance to cluster the 8 data points into $k=3$ clusters. The coordinates of the data points are:
\begin{align*}
    x^{(1)} & = (2,8),  \ x^{(2)} = (2,5), \ x^{(3)} = (1,2), \ x^{(4)} = (5,8), \\
    x^{(5)} & = (7,3),  \ x^{(6)} = (6,4), \ x^{(7)} = (8,4), \ x^{(8)} = (4,7).
\end{align*}
Suppose that initially the first cluster centers is $x^{(1)}$. \\
{\color{blue} To ensure consistent results, please use random numbers in the order shown in the table below. When selecting a center, arrange it in ascending order of sequence number. For example, when the normalized weights of $5$ nodes are $0.2$, $0.1$, $0.3$, $0.3$, and $0.1$, if the random number is $0.3$, the selected node is the third one. Note that you don't necessarily need to use all of them.}
\begin{table*}[h]
    \centering
    \begin{tabular}{|c|c|c|c|c|}
    \hline
    0.6 & 0.2 & 0.5 & 0.9 & 0.3 \\
    \hline
    \end{tabular}
\end{table*}

\begin{itemize}
\item[(a)] Perform the $k$-means++ algorithm to initialize other centers and report the coordinates of the resulting centroids. ~\defpoints{5}
\item[(b)] Calculate the loss function
$$Q(r,c) = \dfrac{1}{n} \sum_{i=1}^n \sum_{j=1}^K r_{ij}\left\|x^{(i)} - c_j\right\|_2^2$$
where $r_{ij} = 1$ if $x^{(i)}$ belongs to the $j$-th cluster and 0 otherwise. ~\defpoints{5}
\item[(c)] How many more iterations after initialization are needed to converge? ~\defpoints{3} Calculate the loss after it converged. ~\defpoints{2}
\end{itemize}

\solution












\newpage