\item \defpoints{15} [Performing PCA by Hand]

Let's do principal components analysis (PCA)! Consider this sample of six points $X_i \in \mathbb{R}^2$.
$$\left\{
\left[\begin{array}{l} 0 \\ 0 \end{array}\right],
\left[\begin{array}{l} 0 \\ 1 \end{array}\right],
\left[\begin{array}{l} 1 \\ 0 \end{array}\right],
\left[\begin{array}{l} 1 \\ 2 \end{array}\right],
\left[\begin{array}{l} 2 \\ 1 \end{array}\right],
\left[\begin{array}{l} 2 \\ 2 \end{array}\right]\right\}$$

(a) Compute the mean of the sample points and write the centered design matrix $\dot{X}$. \defpoints{4}
(Hint: The sample mean is by subtracting the mean from each sample.)

(b) Find all the principal components of this sample. Write them as unit vectors. \defpoints{5}
(Hint: The principal components of our dataset are the eigenvectors of the matrix $\dot{X}^{\top} \dot{X}=
$. The characteristic polynomial of this symmetric matrix is $\det\left(\lambda I-\dot{X}^{\top} \dot{X}\right)$.)

(c) Which of those two principal components would be preferred if you use only one? \defpoints{2} \\
What information does the PCA algorithm use to decide that one principal components is better than another? \defpoints{2} \\
From an optimization point of view, why do we prefer that one? \defpoints{2}

\solution










\newpage